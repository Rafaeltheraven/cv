%-----------------------------------------------------------------------------------------------------------------------------------------------%
%	The MIT License (MIT)
%
%	Copyright (c) 2021 Jitin Nair
%
%	Permission is hereby granted, free of charge, to any person obtaining a copy
%	of this software and associated documentation files (the "Software"), to deal
%	in the Software without restriction, including without limitation the rights
%	to use, copy, modify, merge, publish, distribute, sublicense, and/or sell
%	copies of the Software, and to permit persons to whom the Software is
%	furnished to do so, subject to the following conditions:
%	
%	THE SOFTWARE IS PROVIDED "AS IS", WITHOUT WARRANTY OF ANY KIND, EXPRESS OR
%	IMPLIED, INCLUDING BUT NOT LIMITED TO THE WARRANTIES OF MERCHANTABILITY,
%	FITNESS FOR A PARTICULAR PURPOSE AND NONINFRINGEMENT. IN NO EVENT SHALL THE
%	AUTHORS OR COPYRIGHT HOLDERS BE LIABLE FOR ANY CLAIM, DAMAGES OR OTHER
%	LIABILITY, WHETHER IN AN ACTION OF CONTRACT, TORT OR OTHERWISE, ARISING FROM,
%	OUT OF OR IN CONNECTION WITH THE SOFTWARE OR THE USE OR OTHER DEALINGS IN
%	THE SOFTWARE.
%	
%
%-----------------------------------------------------------------------------------------------------------------------------------------------%

%----------------------------------------------------------------------------------------
%	DOCUMENT DEFINITION
%----------------------------------------------------------------------------------------

% article class because we want to fully customize the page and not use a cv template
\documentclass[a4paper,12pt]{article}

%----------------------------------------------------------------------------------------
%	FONT
%----------------------------------------------------------------------------------------

% % fontspec allows you to use TTF/OTF fonts directly
% \usepackage{fontspec}
% \defaultfontfeatures{Ligatures=TeX}

% % modified for ShareLaTeX use
% \setmainfont[
% SmallCapsFont = Fontin-SmallCaps.otf,
% BoldFont = Fontin-Bold.otf,
% ItalicFont = Fontin-Italic.otf
% ]
% {Fontin.otf}

%----------------------------------------------------------------------------------------
%	PACKAGES
%----------------------------------------------------------------------------------------
\usepackage{url}
\usepackage{parskip} 	

%other packages for formatting
\RequirePackage{color}
\RequirePackage{graphicx}
\usepackage[usenames,dvipsnames]{xcolor}
\usepackage[scale=0.9]{geometry}

%tabularx environment
\usepackage{tabularx}

%for lists within experience section
\usepackage{enumitem}

% centered version of 'X' col. type
\newcolumntype{C}{>{\centering\arraybackslash}X} 

%to prevent spillover of tabular into next pages
\usepackage{supertabular}
\usepackage{tabularx}
\newlength{\fullcollw}
\setlength{\fullcollw}{0.47\textwidth}

%custom \section
\usepackage{titlesec}				
\usepackage{multicol}
\usepackage{multirow}

%CV Sections inspired by: 
%http://stefano.italians.nl/archives/26
\titleformat{\section}{\Large\scshape\raggedright}{}{0em}{}[\titlerule]
\titlespacing{\section}{0pt}{10pt}{10pt}

%for publications
\usepackage[style=authoryear,sorting=ynt, maxbibnames=2]{biblatex}

%Setup hyperref package, and colours for links
\usepackage[unicode, draft=false]{hyperref}
\definecolor{linkcolour}{rgb}{0,0.2,0.6}
\hypersetup{colorlinks,breaklinks,urlcolor=linkcolour,linkcolor=linkcolour}
\addbibresource{citations.bib}
\setlength\bibitemsep{1em}

%for social icons
\usepackage{fontawesome5}

%debug page outer frames
%\usepackage{showframe}


% job listing environments
\newenvironment{jobshort}[2]
    {
    \begin{tabularx}{\linewidth}{@{}l X r@{}}
    \textbf{#1} & \hfill &  #2 \\[3.75pt]
    \end{tabularx}
    }
    {
    }

\newenvironment{joblong}[2]
    {
    \begin{tabularx}{\linewidth}{@{}l X r@{}}
    \textbf{#1} & \hfill &  #2 \\[3.75pt]
    \end{tabularx}
    \begin{minipage}[t]{\linewidth}
    \begin{itemize}[nosep,after=\strut, leftmargin=1em, itemsep=3pt,label=--]
    }
    {
    \end{itemize}
    \end{minipage}    
    }



%----------------------------------------------------------------------------------------
%	BEGIN DOCUMENT
%----------------------------------------------------------------------------------------
\begin{document}

% non-numbered pages
\pagestyle{empty} 

%----------------------------------------------------------------------------------------
%	TITLE
%----------------------------------------------------------------------------------------

% \begin{tabularx}{\linewidth}{ @{}X X@{} }
% \huge{Your Name}\vspace{2pt} & \hfill \emoji{incoming-envelope} email@email.com \\
% \raisebox{-0.05\height}\faGithub\ username \ | \
% \raisebox{-0.00\height}\faLinkedin\ username \ | \ \raisebox{-0.05\height}\faGlobe \ mysite.com  & \hfill \emoji{calling} number
% \end{tabularx}

\begin{tabularx}{\linewidth}{@{} C @{}}
\Huge{Rafael Malach Dulfer} \\[7.5pt]
\href{https://github.com/rafaeltheraven}{\raisebox{-0.05\height}\faGithub\ Rafaeltheraven} \ $|$ \ 
\href{https://linkedin.com/in/rafael-dulfer-b60a32111/}{\raisebox{-0.05\height}\faLinkedin\ Rafael Dulfer} \ $|$ \ 
\href{https://dulfer.be}{\raisebox{-0.05\height}\faGlobe \ dulfer.be} \ $|$ \ 
\href{mailto:rafael@dulfer.be}{\raisebox{-0.05\height}\faEnvelope \ rafael@dulfer.be} \ $|$ \ 
\href{tel:+31616935305}{\raisebox{-0.05\height}\faMobile \ +316.169.353.05} \\
\end{tabularx}

%----------------------------------------------------------------------------------------
% EXPERIENCE SECTIONS
%----------------------------------------------------------------------------------------

%Interests/ Keywords/ Summary
% \section{Summary}
% This CV is automatically generated and deployed using the \href{https://github.com/jitinnair1/autoCV}{autoCV} template along with GitHub Actions such that a new version of the CV is compiled, published and ready for use when the cv.tex file is updated. For details, \href{https://github.com/jitinnair1/autoCV}{click here}.

%Experience
\section{Werk Ervaring}

\begin{joblong}{Web Developer - Netherlands Society of Cinematography}{Jan 2023 - Heden}
& \hfill \href{https://cinematography.nl}{cinematography.nl} \\[3.75]
\item Website overgenomen van vorige developer.
\item Backend ge-streamlined om makkelijker functionaliteit toe te voegen.
\item Functionaliteit toegevoegd op aanvraag van client.
\item Versnellen van inladen sommige paginas door betere architectuur.
\item Security issues gevonden en opgelost.
\item PHP-based.
\end{joblong}

\begin{jobshort}{Software Development - BRBA}{Feb 2021 - Jul 2021}
Voor de periode Februari - July 2021 hield ik mij in opdracht van Underdark bezig met het \href{https://www.ronroozendaal.nl/blog/2021/05/een-nieuw-registratiesysteem-in-een-paar-weken-tijd}{BRBA} vaccinatie registratie project voor het Nederlandse Ministerie van
Volksgezondheid, Welzijn en Sport (MinVWS). Ik was hierbij vooral bezig met validatie+encryptie, integration testing en het streamlinen van het build proces. Ook vervulde ik verschillende losse taken waar nodig.
\end{jobshort}

\begin{jobshort}{Software Development - Serious Game Automatisering}{Jun 2019 - Aug 2019}
In opdracht van De Winter Information Solutions heb ik een dedicated interface ontworpen om het ``hosten'' van hun serious game voor de Vereniging van Nederlandse Gemeenten gedeeltelijk te automatiseren. 
\end{jobshort}

\begin{joblong}{Web Developer - Anubis}{Jan 2021 - Apr 2022}
\item Online examen systeem voor de Universiteit Twente.
\item Ontstaan vanuit Universiteits project.
\item In gebruik door de vakken Network Systems en Web Science.
\item Zelf-ontworpen Markdown dialect + parser om examens te definiëren. 
\item 100+ leerlingen die tegelijk inladen.
\item PHP-based.
\end{joblong}
  
%Projects
\section{Projecten}

\begin{tabularx}{\linewidth}{ @{}l r@{} }

\textbf{Mr Kite's Maginificent Kut-Up Machine} & \hfill \href{https://mkmkm.thebias.nl}{Stream} \\[3.75pt]
\begin{enumerate}
    \item Een oneindige audio stream die een grote database aan muziek door elkaar husselt, vervormd, samenvoegd en in korte fragmenten afspeeld.
    \item Geschreven in Python + QT voor de \href{https://github.com/Rafaeltheraven/Mr-Kite-s-Magnificent-Kut-up-Machine}{GUI} versie.
\end{enumerate} \\

\textbf{Nethacklang} \\
\begin{enumerate}
    \item Compiler voor een zelf-ontworpen ``taal'' geinspireerd door het spel Nethack waarin de developer een level moet ontwerpen om input/output te kunnen doen.
    \item Geschreven in Haskell.
\end{enumerate} \\

\textbf{WAGon} & \hfill \href{https://wagon.thebias.nl/}{Documentatie} \\[3.75pt]
\begin{enumerate}
    \item Master Thesis Project.
    \item DSL om Weighted Attribute Grammars in te definiëren.
    \item Ecosysteem om met deze DSL te kunnen werken.
    \item GLL-based parser generator als proof-of-concept voor de DSL en het ecosysteem.
    \item Geschreven in Rust.
\end{enumerate} \\

\textbf{The Propaganda Machine} & \hfill \href{https://ieee-cog.org/2021/assets/papers/paper_231.pdf}{Paper}\cite{dulfer-propaganda-2021} \\[3.75pt]
\begin{enumerate}
    \item Bachelor Thesis Project.
    \item Schrijft automatische propaganda gebaseerd op gespeelde potjes Risk.
    \item Onderzoek naar effectiviteit van simpel automatisch gegenereerde propaganda.
    \item Geschreven in Python + Java.
\end{enumerate} \\

\end{tabularx}

%----------------------------------------------------------------------------------------
%	EDUCATION
%----------------------------------------------------------------------------------------
\section{Opleiding}
\begin{tabularx}{\linewidth}{@{}l X@{}}	
2021 - Heden & Msc, Software Technology, \textbf{Universiteit Twente} \\

2017 - 2012 & Bsc, Technical Computer Science, \textbf{Universiteit Twente} \\ 

2019 - 2020 & Minor New Media Language, \textbf{Kaunas University of Technlogy} \\

2018 - 2019 & Bsc of Honours, Philosophy, \textbf{Universiteit Twente} \\

2014 - 2017 & IB English A Language \& Literature, \textbf{Het Berlage Lyceum} \\

2017 - VWO E\&M, \textbf{Het Berlage Lyceum} \\

2015 & Summer Course Computer Science, \textbf{Oxford University} \\

2014 & Junior TTO Certificate, \textbf{Cambridge Education Group} \\
\end{tabularx}

%----------------------------------------------------------------------------------------
%	PUBLICATIONS
%----------------------------------------------------------------------------------------
\section{Publicaties}
\begin{refsection}[citations.bib]
\nocite{*}
\printbibliography[heading=none]
\end{refsection}

%----------------------------------------------------------------------------------------
%	SKILLS
%----------------------------------------------------------------------------------------
\section{Vaardigheden}
\begin{enumerate}
    \item Programmeertalen:
        \begin{enumerate}
            \item Python
            \item PHP
            \item Rust
            \item Java
            \item Haskell
            \item JavaScript
        \end{enumerate}
    \item Reguliere Talen:
        \begin{enumerate}
            \item Nederlands
            \item Engels
        \end{enumerate}
    \item Linux
    \item Server Hosting
\end{enumerate}

\vfill
\center{\footnotesize Last updated: \today}

\end{document}
